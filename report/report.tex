% !TeX spellcheck = fr
\documentclass[12pt,a4paper]{article}

\usepackage[utf8]{inputenc}
\usepackage{mathptmx}
\usepackage{helvet}
\renewcommand{\familydefault}{\sfdefault}
\usepackage{graphicx}
\usepackage{hyperref}

\usepackage[left=2.5cm, right=2.5cm, top=2.5cm]{geometry}

\title{\huge{\textbf{Rapport de projet 150h}}}

\author{
    Guillaume BOEHM \\
    Master 2 I3D\\
    Université de Strasbourg\\\\
	Projet encadré par Pierre Kraemer}
\date{\today}

%-----------------------------------------------------------------------------------------
% The start of the document
%-----------------------------------------------------------------------------------------
\begin{document}

\maketitle
\thispagestyle{empty}

\clearpage

\thispagestyle{empty}
\setcounter{page}{0}
\tableofcontents
\clearpage

\addcontentsline{toc}{section}{\nameref{intro}}
\section*{Introduction}\label{intro} % The start of a new section

Dans le cadre de mon projet 150 de master, j'ai été amené à travailler avec Pierre Kraemer sur le projet CGoGN du laboratoire ICube. CGoGN est une librairie C++ de modélisation géométrique utilisant des cartes combinatoires, un puissant modèles de représentation topologique. Cette librairie possède dors et déjà de nombreux outils de parcours et modification des maillages tels que des parcours sur la totalité du maillage par type de cellules, ou des algorithmes de simplification de maillage. CGoGN utilise de la métaprogrammation avec des patrons afin d'autoriser l'utilisation de modèles différents.\\
Dans ce contexte j'ai du prendre en main la librairie et implémenter l'algorithme de simplification de maillage topstoc\cite{BA:2009:MSS} en l'adaptant au modèle des cartes combinatoires. L'algorithme topstoc utilise une sélection stochastique des sommets du maillage à conserver basée sur leur courbure locale ainsi qu'un partitionnement topologique pour reconstituer le maillage.\\ Par la suite je vais développer mon implémentation de l'algorithme dans Section \ref{implementation}, présenter les résultats obtenus dans Section \ref{resultats} et finir avec mon ressentit du projet dans Section \ref{conclusions}.

\clearpage

\section{Implémentation}\label{implementation}

La majorité des fonctions utilitaires de CGoGN sont des fonctions globales rangées dans des espaces de noms, j'ai donc implémenté l'algorithme topstoc dans une fonction \verb|topstoc| sous l'espace de noms \verb|cgogn::modeling|. Le module \verb|SurfaceModeling|, par le biais duquel il est possible d'effectuer différentes opérations de modification de maillage, permet ensuite d'appliquer la fonction \verb|topstoc| à un maillage.\\\\
L'algorithme implémenté peut être séparé en trois parties détaillées ci-dessous : la sélection des sommets, le partitionnement topologique puis la reconstruction du maillage.

\subsection{Sélection des sommets}\label{sampling}

Ma première implémentation de la sélection des sommets me servait de bouchon et ne faisait que sélectionner les $k$ premiers sommets stockés dans le maillage, avec $k$ le nombre de sommets voulus. Par la suite j'ai implémenté une variante de la méthode amenée dans \cite{BA:2009:MSS}.\\
\`A chaque sommet $v$ du maillage est associé une valeur caractéristique $x_v$ pour guider la sélection, en l'occurrence les information de courbure autour du sommet :

\[ x_v = \frac{\sum_{u \in N_v}\sigma(\textbf{n}_v^\texttt{T}\textbf{n}_u)}{|N_v|} \]

\noindent Avec $N_v$ les sommets du 1-ring de $v$, $\textbf{n}_v$ le vecteur normal de $v$ et $\sigma$ une fonction décroissante monotone sur $[-1,1] \rightarrow [0,1]$, pour laquelle j'ai utilisé la fonction conseillée dans l'article :

\[ \sigma(d) = (1-d)/2 \]

\clearpage

\noindent Ces valeurs caractéristiques sont ensuite utilisées dans une fonction de distribution $\mathcal{P}$ afin de conserver le sommet $v$ si $r < \mathcal{P}(x_v)$, avec $r$ une variable aléatoire continue sur $[0,1]$.\\ Afin d'éviter de laisser les sommets de courbure nulle à une probabilité de 0 et se retrouver avec un sous-échantillonnage des zones plates, la formule suivante est utilisée pour la fonction de distribution :

\Large{\[ \mathcal{P}(x) = \frac{\left(1+\alpha(\frac{x}{\bar{\{x_v\}}}-1)\right)}{X_v} \]}

\normalsize

\noindent Avec $\alpha$ un coefficient d'adaptabilité, pour mon projet j'ai utilisé $\alpha = 2/3$, $\bar{\{x_v\}}$ la moyenne des $x_v$ de $V$, et $X_v$ la plus grande valeur de $x_v$ sur $V$.\\
Cette formule s'éloigne légèrement de la formule proposée dans \cite{BA:2009:MSS} :

\Large{\[ \mathcal{P}(x) = k \left( 1+\alpha(\frac{x}{\bar{\{x_v\}}}-1) \right)  
\]}

\normalsize

\noindent Qui ne résulte pas en une fonction $\mathcal{P} : [0,1] \rightarrow [0,1]$.\\\\

\noindent On se retrouve donc avec $\tilde{k} \approx k$ sommets à conserver pour le nouveau maillage, la prochaine étape est de décider comment relier les sommets sélectionnés.

\subsection{Partitionnement topologique}\label{clustering}

Une fois les sommets sélectionnés il faut trouver les triangles qui permettront de conserver la topologie de l'objet, pour cela 

\subsection{Reconstruction du maillage}\label{reconstruction}



\section{Résultats}\label{resultats}


\section{Conclusions}\label{conclusions}


\bibliographystyle{abbrv}
\bibliography{library}

\end{document}